\documentclass{article}
\usepackage{fullpage}
\usepackage{harvard}
\usepackage{parskip}
\usepackage{comment}
\usepackage{graphicx}
\usepackage{url}
\usepackage{todonotes}
\usepackage{geometry}
\usepackage{helvet}

\geometry{a4paper, portrait, margin=2cm}
\setcounter{secnumdepth}{4}

\renewcommand{\familydefault}{\sfdefault}
\newcommand{\myparagraph}[1]{\paragraph{#1}\mbox{}\\}
\linespread{1}

\begin{document}
\title{Towards a Model for Building Well-Rounded Brands within Digital Games}
\author{Pieter Joubert}
\date{\today}
\maketitle 

\tableofcontents
\listoffigures

\section{Introduction}

\subsection{Contextualisation}

\subsection{Rationale}

\subsection{Problem statement}

\subsection{Research Goal}

\subsection{Research Questions}

\section{Preliminary Literature Review}

\subsection{Game Development/Game Mechanics}

\subsection{Brand Identity}

\subsection{Branded Entertainment}

\subsection{Similar Research}

\section{Research Design and Methodology}

\subsection{Conceptual approach}

qualitative (or maybe mixed methods)

\subsection{Research Design}
Inductive
Explanatory/Action Research
Artefact Analysis + Action Research + Surveys
Multi-disciplinary Design

\subsection{Research Plan}

\subsection{Population and Sampling}

\subsection{Data Collection}

\subsection{Data Analysis}

\section{Proposed Chapter Outline}

\begin{itemize}
    \item \emph{Introduction} - An Introduction to the topic as well as a discussion of the structure of the dissertation.
    \item \emph{Background} - An in-depth discussion of the topic in hand, as well as a detailed discussion of the problem statement, research questions, and the expected contribution.
    \item \emph{Literature Review} - A general literature review as well as two Structured Literature Reviews, including a brief methodological explanation, the reviews themselves, and finally, the analysis and conclusion of the reviews.
    \begin{itemize}
    \item \emph{Data Collection - Taxonomy} - A detailed explanation of the process followed in developing the Brands in Digital Games Taxonomy.
    \item \emph{Data Analysis - Taxonomy} - A detailed analysis of the data collected related to developing the Brands in Digital Games Taxonomy.
    \item \emph{Data Collection - Game Development Model} - A detailed explanation of the process followed in developing a generalised and simplified Game Development Model.
    \item \emph{Data Analysis - Game Development Model} - A detailed analysis of the data collected related to developing the generalised and simplified Game Development Model.
    \end{itemize}
    \item \emph{Methodology} - A detailed explanation of the methodology that will be followed in the dissertation, including Conceptual approach, research design, research plan, data collection and data analysis.
    \item \emph{Action Research - Problem Identification Stage} - A detailed explanation of Identifying a problem that can be solved with a Well Rounded Brand within a Digital Game
    \item \emph{Action Research - Research Stage} - Developing a Model for Building Well Rounded Brands in Digital Games based on results of the Structured Literature Reviews completed earlier, as well as existing Brand Identity Models.
    \item \emph{Action Research - Development Stage} - Developing a Digital Game that includes a Well-Rounded Brand, based on the Model developed in the previous stage.
    \item \emph{Action Research - Evaluation} - Testing the Digital Game to evaluate if the Brand Identity fully came across by playing the Digital Game
    \item \emph{Results} - Detailed explanation of the final results of the Action Research Process.
    \item \emph{Conclusion and Future Research} - Comprehensive Conclusion of the work done as well as identification of possible future research.

\end{itemize}
\section{Ethical Considerations}

The Digital Game that will be developed would need to be evaluated by participants who have used it. As some of these participants might be Vega Students, ethical clearance from the IIE will be required.

\section{Original Contribution to Scientific Knowledge}

Besides the full contribution detailed below there are also a number of smaller contributions, built via Structured Literature Reviews, that could be publishable as conference or journal papers:
\begin{itemize}
    \item A Brands in Digital Games Taxonomy.
    \item A Generalised and Simplified Game Development Model.
\end{itemize}

Development of a Model for building or immigrating a Brand int a Digital Game that is Well-Rounded within the context of the Game in question.


\bibliographystyle{myharvard}
\bibliography{../PHD_literature}

\end{document}
