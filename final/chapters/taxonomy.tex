Taxonomy
\begin{itemize}
    \item Narrative Brands
    \item Aesthetic Brands
    \item Mechanical Brands
    \item Parody/Facsimile Brands
    \item Real world brands 
\end{itemize}

\section{Narrative Brands}

\subsection{Description}

A Narrative Brand is a in game brand that is crucial to the narrative of the game. This Brand can be antagonistic, with the protagonist of the game facing off against the corporation, organisation or team represented by this Brand.

The Narrative Brand could also be on the side of the protagonist, with the protagonist either being a member of this brand, or being supported in some fashion by this brand.

Finally the Narrative Brand could switch between the role of protagonist and antagonist as the narrative of the game progresses.

\subsection{Examples}

Abstergo in Assassin's Creed. (Fan made website: Abstergo.org)
Hyperion in Borderlans 2.

\subsection{Brand Identity}

\section{Aesthetic Brands}

\subsection{Description}

Aesthetic Brands provide background colour to a game world. They fill out the game world providing context to the world. Aesthetic Brands are often only visible within the game as in terms of iconography, billboards, posters and similar visual representations.

The player rarely interacts directly with the brand.
\subsection{Examples}

BurritoXXL in Cyberpunk 2077 (could also be a Facsimile Brand of the XXL Burrito of Taxo Bell)

\subsection{Brand Identity}

\section{Mechanical Brands}

\subsection{Description}

Mechanical Brands are in-game brands that have a mechanical effect on game play. Items that are produced by a certain brand in game might have a certain type of effect or ability.

\subsection{Examples}

Ravnica Guilds from Magic the Gathering.
The Corporations from Borderlands.

\subsection{Brand Identity}

\section{Parody/Facsimile Brands}

\subsection{Description}

A Parody or Facsimile Brand is an in-game brand that is a close copy or parody of an existing real world brand. This is either done for comedic effect or to trade-on the brand recognition of the real-world brand being copied.

\subsection{Examples}

eCola in GTA V
\subsection{Brand Identity}

\section{Real world brands }

\subsection{Description}

Real-world brands in games are examples of actual brands that have been copied over into the game. These Brands generally have some kind of arrangement with the game developers, either to provide permission for the brand to be deployed within the game, or to provide compensation to the game developer for the exposure provided by the game.

\subsection{Examples}
Ford F150-RL in Rocket League
Keanu Reeves' Motorcyle Brand in Cyberpunk 2077
\subsection{Brand Identity}