\documentclass[10pt,twoside]{article}
\usepackage[T1]{fontenc}
\usepackage{harvard}
\usepackage{url}
\pagestyle{plain}

\def\all{all}
\ifx\files\all \typeout{Including all files.} \else %\typeout{Including only \files.} \includeonly{\files} \fi


\title{PhD Short Proposal: Brand Identity, Building and Valuation of Synthetic Brands in Digital Games}
\author{Pieter Joubert \\
	Vega School Bordeaux \\
	}

\date{\today}
% Hint: \title{what ever}, \author{who care} and \date{when ever} could stand 
% before or after the \begin{document} command 
% BUT the \maketitle command MUST come AFTER the \begin{document} command! 
\begin{document}

\maketitle

\section{Introduction}

In this proposal I would like to outline the process I intend to follow for my PhD research, from the Literature Review, the Methodology, the Data Analysis, finally to the Expected Results. 

In the PhD I intend to look at the Building, Identity and Valuation of \emph{Synthetic Brands} in Digital Games. In this context \emph{Synthetic Brands} are defined as brands that have been created entirely for the purpose of entertainment, in various media such as books, television series, movies or digital games.

From this exploration of the Building, Identity and Valuation of \emph{Synthetic Brands} the following contributions will be produced: a deeper understanding of the use of Brands within Digital Games and Digital Game Development, insight into Brands in an abstract context rather than real world Brands that have grown organically, and finally, how to apply existing Identity and Valuation approaches and toolkits to non-traditional brands.

\section{Literature}

The literature that will be used to underpin the research will focus on two main areas of scholarship: Game Development and Branding. The literature on Branding will be further subdivided into sections on Brand Identity, Brand Building and Brand Valuation, as not only do these three areas of research often work hand-in-hand but they also form the pillars of the eventual research outcomes of the proposed dissertation.

\subsection{Game Development}

The literature section regarding Game Development will draw from a number of sources but primarily from publications such as: The Art of Game Design: A Book of Lenses \cite{schell2008art}, Fundamentals of Game Design \cite{adams2014fundamentals}, and others.


\subsection{Brand Identity and Brand Building}

The section on Brand building and Brand Identity will be informed by the Work of Aacker \cite{aaker2012brand}, Enslin \cite{cook2010healthy} on Brand Identity and Brand Building, as well as many otehrs.

\subsection{Brand Valuation and Brand Equity}

The literature on Brand Valuation and Brand Equity will be informed by the work of Keller \cite{keller1998effects} adn the toolkits that have been built upon this model, e.g. The \emph{Interbrand}, \emph{Brand Finance} and \emph{Brandz} toolkits. 

\section{Methodology}

This dissertation will primarily employ a Qualitative research approach, with a number of additional methodologies to help triangulate the findings. These additional approaches will the Artifact Analysis and Virtual Ethnography, and if necessary Expert Interviews.

\subsection{Taxonomy Development}

Alongside the approaches mentioned above, the author will also develop a Taxonomy of Brands within digital games (including "Real World" brands) to assist in defining the types of Brands that can occur within Digital Games.

\subsection{Artifact Analysis}

A large part of the proposed research will be carried out using Artifact Analysis, to analyse various Digital Games and the Brands that exist within them. This methodology will used primarily to help determine the various Brand Identities of the \emph{Synthetic Brands} in question.

\subsection{Virtual Ethnography}

In order to collect data to assist in the Valuation of these \emph{Synthetic Brands}, Virtual Ethnography will be carried out amongst online gaming communities that play the games in question.

\subsection{Expert Interviews}

Finally, if feasible, the author would conduct expert interviews with the developers and designers of the brands within the games in question. These interviews would be used to shore up any missing data within the creation of the Brand Identities of the brands in question.

\section{Ethics}

While the majority of the research will involve Artifact Analysis and thus limited Ethical implications, the use of Virtual Ethnography and Expert Interviews will require ethical clearance.

\section{Data Analysis}

The data will be analysed, using the various models of Brand Identity, Build and Valuation, as well as Game Development, primarily using thematic analysis. This analysis will lead to the development of a model that can be used for building \emph{Synthetic Brands}.

\section{Expected Results}

 The expected results for this research will be:
\begin{enumerate}
	\item A Taxonomy of Brands within Digital Games
	\item A deeper understanding of Brand Identity and Brand Equity
	\item A development model ofr building Synthetic Brands for Digital Games
\end{enumerate}

\section{Conclusion}

This short proposal outlined the primary scope of my proposed research(\emph{Synthetic Brands} in Digital Games), the proposed research approach (Qualitative), the various primary sources of Literature and the Expected Results.

\bibliographystyle{myharvard}
\bibliography{PHD_literature}

\end{document}